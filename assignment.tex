\documentclass[a4paper,10pt,twoside]{article}
\def\theauthor{Ture Teknolog} % TODO: stoppa in ditt namn här
\def\homeworknumber{HT1} % TODO: stoppa in vilken hemläxa det är här
\def\coursename{Introduktion till datalogi} % TODO: stoppan in kursnamn här
\def\course{DD1341} % TODO: stoppa in kurskod här
\def\thedate{\today} % Byt ut om du vill ha annat datum på inlämningen

\usepackage[inner=3cm,top=3cm,outer=2cm,bottom=3cm]{geometry} %Det här fixar så att den har en bra bredd och att det är samma bredd på alla sidor
\usepackage[T1]{fontenc}
\usepackage[english]{babel}
\usepackage[utf8]{inputenc}
\usepackage{listings}
\usepackage{amssymb}
\usepackage{fancyhdr}
\usepackage{fancyvrb}
\usepackage{graphicx}
\usepackage{xcolor}
\definecolor{dark-blue}{rgb}{0, 0, 0.6}
\usepackage{hyperref}
% Nedanstående funkar inte i praktiken, kanske någon annan kan klura ut
\hypersetup{
  colorlinks=true, 
  linkcolor=dark-blue,
  urlcolor=dark-blue
}

% Här definierar man färger, de här är standardfärgerna för java, som de syns i exempelvis BlueJ. Iaf enligt xmas, har tagit de från honom
\definecolor{javared}{rgb}{0.6,0,0} % for strings
\definecolor{javagreen}{rgb}{0.25,0.5,0.35} % comments
\definecolor{javapurple}{rgb}{0.5,0,0.35} % keywords
\definecolor{javadocblue}{rgb}{0.25,0.35,0.75} % javadoc

% Här definierar man alltså hur det ska se ut när den fixar koden
% De flesta saker säger sig själva vad de betyder. Finns eventuellt fler saker också.
\usepackage{listings}
\lstset{language=Java,
	basicstyle=\ttfamily,
	%Här anges alltså vilka färger som sak användas. Vill ni inte ha färger, kommentera ut nästföljande fyra rader.
	keywordstyle=\color{javapurple}\bfseries,
	stringstyle=\color{javared},
	commentstyle=\color{javagreen},
	morecomment=[s][\color{javadocblue}]{/**}{*/},
	numbers=left, % Fixar radnumrering i vänstermarginalen
	numberstyle=\footnotesize,
	title=\lstname,
	showstringspaces=false,
	fancyvrb=true,
	extendedchars=true,
	breaklines=true,
	breakatwhitespace=true,
	tabsize=4 %Indenteringsstorlek
}

% Fixar så man kan ha åäö i kodkommentarer
\lstset{literate={ö}{{\"o}}1
	{ä}{{\"a}}1
	{å}{{\aa}}1
}

%Header och footer
\pagestyle{fancy}
\lhead{\course\ -\ \theauthor}
\rhead{Hemläxa \homeworknumber}
\fancyfoot[LE,RO]{\thepage}
\title{Hemläxa \homeworknumber\ - \course\ \coursename}
\date{\thedate}
\author{\theauthor}
\renewcommand{\headrulewidth}{0.4pt}
\renewcommand{\footrulewidth}{0pt}

\begin{document}

\section{Vad är det här?}

Det här är alltså ett exempel på hur man kan lämna in sina INDA-labbar på ett snyggt sätt (så att man får pluspoäng av sin asse). LaTeX-mallen är uppkommenterad med vad det mesta betyder så att du kan ändra den efter eget behag. 

PRO TIP: Byt ut ditt namn och och kod och allt innan du lämnar in det..... 

% På det här sättet referar man till en fil man vill ha in
\section{Inkludering av fil} % Förslagsvis slänger man in en rubrik

På det här sättet kan man inkludera en hel fil, så att du inte behöver hålla på och copypasta in det i LaTeX. Notera att detta exempel förutsätter att nämnda kod-fil ligger i samma mapp som denna .tex-fil. 

\label{sec:Foo_java} % En minirubrik till inkluderingen
\lstinputlisting{Foo.java} % Och sedan anger man vilken fil man vill att den ska hämta in
% Notera här att filen antas ligga i samma mapp som denna LaTeXs-fil

\section{Inkludering av kod direkt i LaTeX}
\subsection{Hello, World!}

Om du nu bara har så lite att du känner att det är onödigt att lägga koden i en separat fil: Fear not! Man behöver inte göra så....

För att skriva ut ``Hello, World!'' i Java så kan man lämpligtvis köra följande funktion.

\begin{lstlisting}
System.out.println("Hello World!");
\end{lstlisting}

\end{document}

