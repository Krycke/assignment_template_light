\documentclass[a4paper,10pt,twoside]{article}

\usepackage[inner=3cm,top=3cm,outer=2cm,bottom=3cm]{geometry} %Det här fixar så att den har en bra bredd och att det är samma bredd på alla sidor
\usepackage[T1]{fontenc}
%\usepackage[english]{babel}
\usepackage[utf8]{inputenc}
\usepackage{listings}
\usepackage{amssymb}
\usepackage{fancyhdr}
\usepackage{fancyvrb}
\usepackage{graphicx}
\usepackage{xcolor}
\definecolor{dark-blue}{rgb}{0, 0, 0.6}
\usepackage{hyperref}
% Nedanstående funkar inte i praktiken, kanske någon annan kan klura ut
\hypersetup{
  colorlinks=true,
  linkcolor=dark-blue,
  urlcolor=dark-blue
}

% Fixar lite config värden.
\def\theauthor{Ture Teknolog} % TODO: stoppa in ditt namn här
\def\homeworknumber{4} % TODO: stoppa in vilken hemläxa det är här
\def\coursename{Introduktion till datalogi} % TODO: stoppan in kursnamn här
\def\course{DD1341} % TODO: stoppa in kurskod här
\def\coursegroup{1} % TODO: stoppa in gruppnummer
\def\courceleader{Peter "Lemming" Boström} % TODO: stoppa in ovningsledare

\def\thedate{\today} % Byt ut om du vill ha annat datum på inlämningen

% Här definierar man färger, de här är standardfärgerna för java, som de syns i exempelvis BlueJ. Iaf enligt xmas, har tagit de från honom
\definecolor{javared}{rgb}{0.6,0,0} % for strings
\definecolor{javagreen}{rgb}{0.25,0.5,0.35} % comments
\definecolor{javapurple}{rgb}{0.5,0,0.35} % keywords
\definecolor{javadocblue}{rgb}{0.25,0.35,0.75} % javadoc

% Här definierar man alltså hur det ska se ut när den fixar koden
% De flesta saker säger sig själva vad de betyder. Finns eventuellt fler saker också.
\usepackage{listings}
\lstset{language=Java,
	basicstyle=\ttfamily,
	%Här anges alltså vilka färger som sak användas. Vill ni inte ha färger, kommentera ut nästföljande fyra rader.
	keywordstyle=\color{javapurple}\bfseries,
	stringstyle=\color{javared},
	commentstyle=\color{javagreen},
	morecomment=[s][\color{javadocblue}]{/**}{*/},
	numbers=left, % Fixar radnumrering i vänstermarginalen
	numberstyle=\footnotesize,
	title=\lstname,
	showstringspaces=false,
	fancyvrb=true,
	extendedchars=true,
	breaklines=true,
	breakatwhitespace=true,
	tabsize=4 %Indenteringsstorlek
}

% Fixar så man kan ha åäö i kodkommentarer
\lstset{literate={ö}{{\"o}}1
	{ä}{{\"a}}1
	{å}{{\aa}}1
}

% Header och footer
\pagestyle{fancy}\headheight 13pt
\fancyfoot{}
\lhead{\course\ -\ Inlämning \homeworknumber}
\rhead{\theauthor\ -\ \thedate}
\fancyfoot[LE,RO]{\thepage}
\title{Inlämning \homeworknumber\ - \course\ \coursename}
\date{\thedate}
\author{\theauthor}
\renewcommand{\headrulewidth}{0.4pt}
\renewcommand{\footrulewidth}{0pt}

\begin{document}

% Ta bort labbcover + clearpage om du inte vill ha inda-bladet, lägg då till \maketitle exempelvis.
%\textheight=24.0cm
%\textwidth=15.1cm
%
%\topmargin=-1.5cm
%%\headheight=0cm
%\headsep=0.7cm
%\oddsidemargin=0.4cm
%\evensidemargin=0.4cm
%%\columnsep=0.8cm
%\footskip=1.5cm
%\setcounter{secnumdepth}{0}
%\pagestyle{empty}
%%\markboth{}{Namn: \hfill Personnr: \hfill}
%%\markboth{\course inda11}{\course inda11}

%\begin{document}

\thispagestyle{empty}
\section*{\course \ \coursename}
%\section*{\course \ \coursename 2011/2012}

\vspace{10mm}

\subsection*{Uppgift nummer: \texttt{\textmd{\homeworknumber}}}

\vspace{3mm}

\subsection*{Namn: \texttt{\textmd{\theauthor}}}

\vspace{3mm}

\subsection*{Grupp nummer: \texttt{\textmd{\coursegroup}}}

\vspace{3mm}

\subsection*{Övningsledare: \texttt{\textmd{\courceleader}}}


\vspace{10mm}

\begin{tabular}{l}
 \hspace{140mm} \\
\hline \hline
\end{tabular}

\vspace{5mm}

\subsection*{Betyg: ..... \hspace{2mm}  Datum: .............. \hspace{2mm} Rättad av: ........................................}


%\end{document}

\clearpage
\thispagestyle{empty}
\mbox{} % empty page for duplex
\clearpage

\setcounter{page}{1}

% TODO: stoppa in faktiskt innehåll här.

\section{Vad är det här?}
\label{sec:vad_är_det_här}

Det här är alltså ett exempel på hur man kan lämna in sina INDA-labbar på ett snyggt sätt (så att man får guldstjärnor av sin asse). \LaTeX-mallen är förhoppningsvis kommenterad tillräckligt så att man ser vad det mesta betyder så att du kan ändra den efter eget behag.

Kom ihåg att byta ut ditt namn, kod och innehåll innan du lämnar in inlämningen. :)

\section{Källkod}
\label{sec:källkod}

På det här sättet kan man inkludera en hel fil, så att du inte behöver hålla på och kopiera det in i \LaTeX. Notera att detta exempel förutsätter att nämnda kod-fil ligger i samma mapp som denna .tex-fil.

\subsection{Foo.java}
\label{sec:Foo_java} % En label till inkluderingen
\lstinputlisting{Foo.java} % Och sedan anger man vilken fil man vill att den ska hämta in
% Notera här att filen antas ligga i samma mapp som denna LaTeXs-fil

\subsection{Inkludering av kod direkt i \LaTeX}

Ibland ska bara mindre småsaker redovisas, då kan det vara smidigare att lägga koden direkt i detta .tex-dokument som i detta exempel.

För att skriva ut ``Hello, World!'' i Java så kan man lämpligtvis köra följande metodanrop.

\begin{lstlisting}
System.out.println("Hello World!");
\end{lstlisting}

\end{document}

