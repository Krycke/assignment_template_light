\documentclass[a4paper,10pt,twoside]{article}
\usepackage[inner=3cm,top=3cm,outer=2cm,bottom=3cm]{geometry}
\def\theauthor{Ture Teknolog} % TODO: stoppa in ditt namn här
\def\homeworknumber{HT1} % TODO: stoppa in vilken hemläxa det är här
\def\coursename{Introduktion till datalogi} % TODO: stoppan in kursnamn här
\def\course{DD1341} % TODO: stoppa in kurskod här
\def\thedate{\today} % Byt ut om du vill ha annat datum på inlämningen

\usepackage[swedish]{babel}
%\usepackage[english]{babel}
\usepackage[T1]{fontenc}
\usepackage[utf8]{inputenc} % Comment out this if you use xelatex or non-UTF8
%\usepackage{moreverb}
\usepackage{amssymb}
\usepackage{fancyhdr}
\usepackage{fancyvrb}
%\usepackage{algorithmic}
%\usepackage{algorithm}
\usepackage{amssymb}
\usepackage{graphicx}
\usepackage{xcolor}
\definecolor{dark-blue}{rgb}{0, 0, 0.6}
\usepackage{hyperref}
\hypersetup{
  colorlinks=true, 
  linkcolor=dark-blue,
  urlcolor=dark-blue
}
%\usepackage{minted}

\usepackage{listings}
\lstset{language=Java,
	numbers=left,
	numberstyle=\footnotesize,
	title=\lstname,
	showstringspaces=false,
	fancyvrb=true,
	extendedchars=true,
	breaklines=true,
	breakatwhitespace=true,
	tabsize=4
}

\lstset{literate={ö}{{\"o}}1
	{ä}{{\"a}}1
	{å}{{\aa}}1
}

\pagestyle{fancy}\headheight 13pt
\fancyfoot{}
\lhead{\course\ -\ Inlämning \homeworknumber}
\rhead{\theauthor\ -\ \thedate}
\fancyfoot[LE,RO]{\thepage}
\title{Inlämning \homeworknumber\ - \course\ \coursename}
\date{\thedate}
\author{\theauthor}

\begin{document}
% Ta bort labbcover + clearpage om du inte vill ha inda-bladet, lägg då till \maketitle
%\textheight=24.0cm
%\textwidth=15.1cm
%
%\topmargin=-1.5cm
%%\headheight=0cm
%\headsep=0.7cm
%\oddsidemargin=0.4cm
%\evensidemargin=0.4cm
%%\columnsep=0.8cm
%\footskip=1.5cm
%\setcounter{secnumdepth}{0}
%\pagestyle{empty}
%%\markboth{}{Namn: \hfill Personnr: \hfill}
%%\markboth{\course inda11}{\course inda11}

%\begin{document}

\thispagestyle{empty}
\section*{\course \ \coursename}
%\section*{\course \ \coursename 2011/2012}

\vspace{10mm}

\subsection*{Uppgift nummer: \texttt{\textmd{\homeworknumber}}}

\vspace{3mm}

\subsection*{Namn: \texttt{\textmd{\theauthor}}}

\vspace{3mm}

\subsection*{Grupp nummer: \texttt{\textmd{\coursegroup}}}

\vspace{3mm}

\subsection*{Övningsledare: \texttt{\textmd{\courceleader}}}


\vspace{10mm}

\begin{tabular}{l}
 \hspace{140mm} \\
\hline \hline
\end{tabular}

\vspace{5mm}

\subsection*{Betyg: ..... \hspace{2mm}  Datum: .............. \hspace{2mm} Rättad av: ........................................}


%\end{document}

\clearpage
\thispagestyle{empty}
\mbox{} % empty page for duplex
\clearpage 

\setcounter{page}{1}
% TODO: stoppa in faktiskt innehåll här.

\section{Någon godtycklig sektion}
\label{sec:någon_godtycklig_sektion}
Lorem ipsum dolor sit amet, consectetur adipisicing elit, sed do eiusmod tempor incididunt ut labore et dolore magna aliqua. Ut enim ad minim veniam, quis nostrud exercitation ullamco laboris nisi ut aliquip ex ea commodo consequat. Duis aute irure dolor in reprehenderit in voluptate velit esse cillum dolore eu fugiat nulla pariatur. Excepteur sint occaecat cupidatat non proident, sunt in culpa qui officia deserunt mollit anim id est laborum.

\subsection{En godtycklig undersektion}
\label{sub:en_godtycklig_undersektion}
Lorem ipsum dolor sit amet, consectetur adipisicing elit, sed do eiusmod tempor incididunt ut labore et dolore magna aliqua. Ut enim ad minim veniam, quis nostrud exercitation ullamco laboris nisi ut aliquip ex ea commodo consequat. Duis aute irure dolor in reprehenderit in voluptate velit esse cillum dolore eu fugiat nulla pariatur. Excepteur sint occaecat cupidatat non proident, sunt in culpa qui officia deserunt mollit anim id est laborum.

\subsection{En annan godtycklig undersektion}
\label{sub:en_annan_godtycklig_undersektion}
Lorem ipsum dolor sit amet, consectetur adipisicing elit, sed do eiusmod tempor incididunt ut labore et dolore magna aliqua. Ut enim ad minim veniam, quis nostrud exercitation ullamco laboris nisi ut aliquip ex ea commodo consequat. Duis aute irure dolor in reprehenderit in voluptate velit esse cillum dolore eu fugiat nulla pariatur. Excepteur sint occaecat cupidatat non proident, sunt in culpa qui officia deserunt mollit anim id est laborum.
% subsection en_godtycklig_undersektion (end)
% section någon_godtycklig_sektion (end)


\section{Källkod}
\label{sec:källkod}

\subsection{Foo.java}
\label{sub:foo_java}

Den här klassen är till för att demonstrera line-wrapping och att å, ä respektive ö fungerar bra i kodfiler också.

% skriv in namnet på en javafil för att inkludera den
\lstinputlisting{Foo.java}

\subsection{Hello, World!}

För att skriva ut ``Hello, World!'' i Java så kan man lämpligtvis köra följande metodanrop.

\begin{lstlisting}
System.out.println("Hello World!");
\end{lstlisting}

% subsection foo_java (end)
% section källkod (end)

\end{document}

